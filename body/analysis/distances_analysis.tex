\subsection{Analysis of Distance Estimates}

\subsubsection{Model / Manual Distance Comparison}\label{subsubsec:distance_comparison}

Ideally, each and every manual distance should be joined to its corresponding modelled
distance estimate; however, due to the absence of any frame-position data associated
with the manual annotations, automating this using traditional algorithms is impossible
in circumstances where multiple chimps are captured in a single frame.
This is because when multiple individuals are detected, there is ambiguity in regard to
which distance a given modelled distance should be joined to.
Moreover, approaching this task manually is extremely labour-intensive and was therefore
outside the scope of this project.
As a result, this analysis focuses on distance comparisons associated with frames capturing
only a single individual.

\begin{figure}[htbp]
    \centering
    \vspace{1cm}
    \includegraphics[width=1.01\textwidth]{body/analysis/assets/distance_graphs/averages3}
    \caption{Distance Comparison}
    \label{fig:distance_comparison}
\end{figure}

\clearpage

\begin{figure}[htbp]
    \centering
    \includegraphics[width=1.01\textwidth]{body/analysis/assets/distance_graphs/spread}
    \caption{Spread Comparison}
    \label{fig:spread_comparison}
\end{figure}

How does parameterisation affect the estimates

\clearpage

\subsubsection{Error Analysis}

\begin{table}[htbp]
    \centering
    \caption{Errors}
    \label{tab:distances}
    \begin{tabular}{ccccc}
        \textbf{Method} & \textbf{$\Delta_{average}$ / m} & \textbf{MAE / m} & \textbf{RMSE / m}
        & \textbf{Statistical Difference} \\
        \midrule
        DPT, BBOX & 0.586 & 1.81 & 2.66 & YES \\
        DPT, SEG  & 0.836 & 1.70 & 2.45 & YES \\
        DA, BBOX  & 1.49  & 2.03 & 2.62 & YES \\
        DA, SEG   & 2.80  & 3.00 & 3.52 & YES \\
    \end{tabular}
\end{table}

\clearpage

\begin{figure}[H]
    \centering
    \includegraphics[width=1.01\textwidth]{body/analysis/assets/errors/MAE}
    \caption{Mean Average Error}
    \label{fig:mae}
\end{figure}

\begin{figure}[H]
    \centering
    \includegraphics[width=1.01\textwidth]{body/analysis/assets/errors/RMSE}
    \caption{Root Mean Squared Error}
    \label{fig:rmse}
\end{figure}

\clearpage


\subsubsection{Qualitative Analysis}
depth map diagrams
close/far failure cases, sweet spot

\subsubsection{Effects of Varying Calibration}