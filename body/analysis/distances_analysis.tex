\subsection{Analysis of Distance Estimates}

\subsubsection{Model / Manual Distance Comparison}\label{subsubsec:distance_comparison}

In this section, the precision and accuracy of the distance estimates generated using the
four configurations of the distance estimation pipeline (outlined in
Section~\ref{subsubsec:configuratons}) are evaluated.
In order for these data to be analysed, the estimates are benchmarked against their
corresponding manual distance estimates (supplied with the dataset), thus these estimates
use detection frames originating from the manual sample (Section~\ref{subsubsec:sampling}).

Ideally, each and every manual distance should be joined to its corresponding modelled
distance estimate; however, due to the absence of any frame-position data associated
with the manual annotations, automating this using traditional algorithms is impossible
in circumstances where multiple chimps are captured in a single frame.
This is because when multiple individuals are detected, there is ambiguity in regard to
which distance a given modelled distance should be joined to.
Moreover, approaching this task manually is extremely labour-intensive and was therefore
outside the scope of this project.
As a result, this analysis focuses on distance comparisons associated with frames capturing
only a single individual.

Upon joining the modelled distance estimates to their corresponding manual estimates, the
overall errors of each of the four pipeline configurations were calculated.
Table~\ref{tab:overall_errors} shows the mean average error, root mean squared error,
average difference (i.e., mean($model_i$ - $manual_i$)) and the result of a statistical
difference test (i.e., where p-value < 0.05).
To note, while both the mean average and root mean squared errors give a measure of the
absolute error of the configurations, the average difference does not and is affected by
cancellation between overestimates and underestimates.
Therefore, it is used as a metric to assess whether a given configuration over or
under-predicts relative to the manual distance estimation method, where a positive value
indicates over-prediction while a negative value indicates under-prediction.

The modelled distance estimates were then grouped by their corresponding manual estimates (i.e.,
0.5 m, 1.0 m, \ldots 15 m).
Figure~\ref{fig:distance_comparison} shows the averages of the modelled estimates for each
group while Figure~\ref{fig:spread_comparison} shows each individual modelled estimate for
each group, giving a visualisation of the spread of the estimates.
Table~\ref{tab:regression_gradients} shows the gradients of the fitted regression lines
correlating the averages of all grouped distance estimates with their corresponding manual
distance estimate for each configuration.
These individual errors for each of these groups were then calculated.
Figure~\ref{fig:mae} show the change in mean average error for each group while
Figure~\ref{fig:rmse} shows the change in root mean squared error for each group.

\vspace{1cm}

\begin{table}[htbp]
    \centering
    \caption{Mean average error (MAE), root mean squared error (RMSE), average difference between
    model and manual estimate ($\Delta_{average}$) and result of Wilcoxon signed rank test for
    statistical difference. These data describe all distance estimates for each pipeline
    configuration collectivelly.)}
    \label{tab:overall_errors}
    \begin{tabular}{ccccc}
        \textbf{Method} & \textbf{MAE / m} & \textbf{RMSE / m} & \textbf{$\Delta_{average}$ / m}
        & \textbf{Statistical Difference} \\
        \midrule
        DPT, BBOX & 1.81 & 2.66 & 0.586 & YES \\
        DPT, SEG  & 1.70 & 2.45 & 0.836 & YES \\
        DA, BBOX  & 2.03 & 2.62 & 1.49  & YES \\
        DA, SEG   & 3.00 & 3.52 & 2.80  & YES \\
    \end{tabular}
\end{table}

\clearpage

\begin{figure}[H]
    \centering
    \vspace{1cm}
    \includegraphics[width=1.01\textwidth]{body/analysis/assets/distance_graphs/averages}
    \caption{Graphs showing mean modelled distance estimates mapped to thier corresponding manual
    estimates for the configurations: DPT/bounding box (top-left), DPT/segmentation (top-right),
        DA/bounding box (bottom-left) and DA/segmentation (bottom-right). The blue dotted line
        shows the fitted regression line. The red dashed line shows the ideal (i.e., model=manual).
        the error bars show the 25–75 (green) and 5–95 (purple) percentiles. All distances are in
        units of meters.}
    \label{fig:distance_comparison}
\end{figure}

\vspace{1cm}

\begin{table}[H]
    \centering
    \caption{Gradients of the regression lines (blue dotted lines in Figure~\ref{fig:distance_comparison})
        correlating the averages of all binned distance estimates with their corresponding manual distance
        estimate for each configuration.}
    \label{tab:regression_gradients}
    \begin{tabular}{cc}
        \textbf{Method} & \textbf{Regression Gradient}\\
        \midrule
        DPT, BBOX & 0.59 \\
        DPT, SEG  & 0.84 \\
        DA, BBOX  & 0.49 \\
        DA, SEG   & 0.80 \\
    \end{tabular}
\end{table}

\begin{figure}[p]
    \centering
    \includegraphics[width=1.01\textwidth]{body/analysis/assets/distance_graphs/spread}
    \caption{Graphs showing all modelled distance estimates mapped to thier corresponding manual
        estimates for the configurations: DPT/bounding box (top-left), DPT/segmentation (top-right),
        DA/bounding box (bottom-left) and DA/segmentation (bottom-right). The red dashed line shows
        the ideal (i.e., model=manual). All distances are in units of meters.}
    \label{fig:spread_comparison}
\end{figure}

\clearpage

\begin{figure}[H]
    \vspace{1cm}
    \centering
    \includegraphics[width=1.01\textwidth]{body/analysis/assets/errors/MAE}
    \caption{Mean average error for distance estimates grouped by their corresponding manual
    estimates for each pipeline configuration.}
    \label{fig:mae}

    \vspace{2cm}

    \includegraphics[width=1.01\textwidth]{body/analysis/assets/errors/RMSE}
    \caption{Root mean squared error for distance estimates grouped by their corresponding manual
    estimates for each pipeline configuration.}
    \label{fig:rmse}
\end{figure}

\clearpage

\textbf{Detection Method Effects}

Contrasting the distance estimates of the two detection method using DPT distance estimation,
an increase in both accuracy and precision is observed for the segmentation method compared
to the bounding box method at close distances (i.e., < 2 meters).
For modelled distances corresponding to manual estimates of 0.5 meters, the bounding box
method gives a MAE of 5.74 meters, a RMSE of 7.17 meters and an interquartile range of 7.50
meters while the segmentation method gives a MAE of 1.98 meters, a RMSE of 3.27 meters and
an interquartile range of 1.61 meters.
This is explained by a better detection frame calibration alignment.
Here, only the detection frame pixels defined by the chimpanzee segmentation are masked as
opposed to those within the entire bounding box.
As a result, more pixels corresponding to the transect background which are common to both
the calibration and detection frames are available for depth scale alignment, leading to a
superior calibration of the detection frame depth scale.

This effect is exemplified in Figure~\ref{fig:bbox_vs_seg_close_dpt}, where the bounding box
method results in an effectively failed calibration while the segmentation method succeeds.

\begin{figure}[htbp]
    \centering
    \makebox[0.8\textwidth][c]{
        \includegraphics[width=0.9\textwidth]{body/analysis/assets/depth_maps/close_bbox_dpt}
    }\\[1mm]
    \makebox[0.8\textwidth][c]{
        \includegraphics[width=0.9\textwidth]{body/analysis/assets/depth_maps/close_seg_dpt}
    }
    \caption{Example of DPT depth maps generated using bounding box (top) and segmentation
        (bottom) detection methods at a detection distance of 0.5 meters (manual estimate).
        In this example, the bounding box method gives a distance estimate of 15.0 meters
        while the segmentation method gives a distance estimate of 1.0 meters}
    \label{fig:bbox_vs_seg_close_dpt}
\end{figure}

In contrast, this detection effect does not apply when Depth Anything is used to estimate
distance.
No significant difference is observed in the accuracy and/or precision of the distance
estimates obtained using the two detection methods.
For modelled distances corresponding to manual estimates of 0.5 meters, the bounding box
method gives a MAE of 4.06 meters, a RMSE of 4.19 meters and an interquartile range of 0.95
meters while the segmentation method gives a MAE of 4.35 meters, a RMSE of 4.60 meters and
an interquartile range of 0.80 meters.
This is an expected result given that Depth Anything is a metric depth model.
Unlike with DPT, the scale of the generated depth maps are not aligned to that of the reference
frames meaning that the detection method does not influence this scale.
Therefore, it can be inferred that the difference in distance estimates given by these detection
methods is a result of the different pixel sampling techniques used to calculate the final distance
estimate (i.e., bounding box depth from the pixel corresponding to the 20th percentile depth
and segmentation depth from the depth of the centre-most pixel) rather than differences on depth
scaling.

Figure~\ref{fig:bbox_vs_seg_close_da} shows the corresponding Depth Anything depth maps using both
detection methods for the same example shown in Figure~\ref{fig:bbox_vs_seg_close_dpt}.
It can be seen that both detection methods lead to identical depth maps, with the final detection
distance estimates being very close.

\begin{figure}[htbp]
    \centering
    \makebox[0.8\textwidth][c]{
        \includegraphics[width=0.7\textwidth]{body/analysis/assets/depth_maps/close_bbox_da}
    }\\[1mm]
    \makebox[0.8\textwidth][c]{
        \includegraphics[width=0.7\textwidth]{body/analysis/assets/depth_maps/close_seg_da}
    }
    \caption{Example of Depth Anything depth maps generated using bounding box (top) and segmentation
        (bottom) detection methods at a detection distance of 0.5 meters (manual estimate).
        In this example, the bounding box method gives a distance estimate of 4.11 meters
        while the segmentation method gives a distance estimate of 4.17 meters.}
    \label{fig:bbox_vs_seg_close_da}
\end{figure}

Upon inspection of Table~\ref{tab:regression_gradients}, it can be seen that for both DPT and Depth
Anything models, the gradients of the regression lines (correlating the averages of the binned
distance estimates with the corresponding manual estimates) of the segmentation detection methods
are higher relative to their corresponding bounding box regression gradient and also better
correlated to the ideal.
A possible explanation for this trend is a minimised contribution of occluding pixels during the
final detection distance calculation at medium to long distances when using the segmentation method.
With this particular dataset, it is from these medium distances where detections start to become
occluded by foliage.
When the bounding box method is used, the final distance can be biased towards a lower estimate in
circumstances where a significantly large portion of the bounding box is composed of occluding pixels
that correspond to a significantly shorter distance.
Conversely, a high quality segmentation often avoids this.
This effect is exemplified in Figure~\ref{fig:bbox_vs_seg_occluded}.

\begin{figure}[htbp]
    \centering
    \makebox[0.8\textwidth][c]{
        \includegraphics[width=0.9\textwidth]{body/analysis/assets/depth_maps/occluded_bbox}
    }\\[1mm]
    \makebox[0.8\textwidth][c]{
        \includegraphics[width=0.9\textwidth]{body/analysis/assets/depth_maps/occluded_seg}
    }
    \caption{Example of DPT depth maps generated using bounding box (top) and segmentation
        (bottom) detection methods at a detection distance of 6.5 meters (manual estimate)
        where the detected individual is partially occluded by foliage. Here, the bounding
        box method gave a distance estimate of 4.40 meters while the segmentation method
        gave a distance estimate of 6.96 meters.}
    \label{fig:bbox_vs_seg_occluded}
\end{figure}

If, however, the quality of the segmentation is poor, the distances of occluded detections
may still be underestimated.
Figure~\ref{fig:haze_occluded} highlights a scenario where an individual was correctly detected
but, due to poor detection frame quality, the occluding foliage was instead segmented rather
than the individual.

\begin{figure}[htbp]
    \centering
    \makebox[0.8\textwidth][c]{
        \includegraphics[width=0.9\textwidth]{body/analysis/assets/depth_maps/occluded_haze_bbox}
    }\\[1mm]
    \makebox[0.8\textwidth][c]{
        \includegraphics[width=0.9\textwidth]{body/analysis/assets/depth_maps/occluded_haze_seg}
    }
    \caption{Example of DPT depth maps generated using bounding box (top) and segmentation
        (bottom) detection methods at a detection distance of 4.5 meters (manual estimate)
        where the detected individual is partially occluded by foliage. Here, moisture on the
        camera lense has resulted in a slightly hazy image, leading to a failed segmentation
        of the detected individual. The bounding box method gave an estimated distance of
        2.64 meters while the segmentation method gave a distance estimate of 2.56 meters.}
    \label{fig:haze_occluded}
\end{figure}


Blurry detections frames

djou09

\vspace{5mm}
\textbf{Depth Model Effects}

As seen in Table~\ref{tab:regression_gradients}, the gradients of the regression lines
correlating the averages of the binned DPT distance estimates with respect to their manual
estimates are greater and also closer the ideal (gradient = 1) than that of Depth Anything.
This shows that, overall, DPT is better capturing the true scale of depth in the detection
frames than Depth Anything.
At short to medium distances (i.e., 2–7 meters), Figure~\ref{fig:distance_comparison} shows
that distance estimate averages closely follow a linear relationship, indicating that within
this distance region, both depth models are differentiating relative depth well.
However, Depth Anything generally over-predicts estimated depth in this region while the DPT
estimates are much closer to the ideal.
This trend also holds for the extreme-close distance region for Depth Anything (with both
detection methods) and also DPT using segmentation (DPT with bounding box over-predicts as
previously discussed).

While DPT gives a better measure of depth scale at close distances, the depth maps show that
Depth Anything is superior at differentiating the depth of fine-detail in the detection frames.
This is shown in Figure~\ref{fig:fine_detail}, where the depth maps generated by all
configurations with a common detection frame are shown side-by-side.


\begin{figure}[H]
    \centering
    \makebox[0.8\textwidth][c]{
        \includegraphics[width=0.9\textwidth]{body/analysis/assets/depth_maps/fine_detail/bbox_dpt}
    }\\[1mm]
    \makebox[0.8\textwidth][c]{
        \includegraphics[width=0.9\textwidth]{body/analysis/assets/depth_maps/fine_detail/seg_dpt}
    }\\[1mm]
    \makebox[0.8\textwidth][c]{
        \includegraphics[width=0.7\textwidth]{body/analysis/assets/depth_maps/fine_detail/bbox_da}
    }\\[1mm]
    \makebox[0.8\textwidth][c]{
        \includegraphics[width=0.7\textwidth]{body/analysis/assets/depth_maps/fine_detail/seg_da}
    }
    \caption{Example of depth maps generated using DPT/BBOX (row one), DPT/SEG
        (row two), DA/BBOX (row three) and DA/SEG (row four) detection methods at a detection
        distance of 13.5 meters (manual estimate). The following distance estimates were given:
        DPT/BBOX = 10.0, DPT/SEG = 10.4 meters, DA/BBOX = 7.83 meters, DA/SEG = 13.6 meters.}
    \label{fig:fine_detail}
\end{figure}

The graphs in Figures~\ref{fig:distance_comparison} and~\ref{fig:spread_comparison} show that
for both DPT configurations as well as Depth Anything with bounding box, distance estimates
corresponding to manual estimates of 13.5 meters are significantly under-predicted.
This is not the case, however, for estimates given by Depth Anything using segmentation detection.
This is an interesting effect which is perfectly illustrated in Figure~\ref{fig:fine_detail}.
In this example both of the DPT depth maps fail to capture any significant differentiation in
depth within the small detection region.
As a result, the final distance estimates yield roughly equal values of approximately 10 meters.
In contrast, the Depth Anything depth map (identical for both detection methods) captures
much more detail in this region, therefore enabling the different pixel sampling techniques used
in the detection distance calculation to yield different results.
In the case of the bounding box method, the contribution of close-distance occluding pixels to the
percentile calculation skew the distance estimate, resulting in a value of 7.83 meters.
For the segmentation method, however, this is no such influence since detection distance is
determined by the depth of centre-most point of the segmentation, resulting in a value of 13.6
meters which is aligned closely to the manual estimate of 13.5 meters.
Results such as these are achieved in spite of imperfect segmentation since the most important
factor in these cases, where there is high contrast surrounding the desired pixel sampling region,
is negating the influence of these surrounding pixels.

\vspace{5mm}
\textbf{False Positives and Misses}

\vspace{5mm}
\textbf{Summary}

Errors table

All configurations over-predict

Sweet spot 2–7 meters

Error spike at 7.5 meters

Literature

\clearpage

\subsubsection{Effects of Varying Calibration}