\subsection{Activity Estimation}

Using the computed distance estimates from the previous section, chimpanzee population
density and abundance was estimated.
Estimates were calculated for each of the detector/depth model combinations using a
script adapted from one supplied with the dataset.

\subsubsection{Formatting}

For the distance data to be used for density and abundance estimation, it first must be
formatted and joined with the relevant location and camera trap video metadata as to
satisfy the script's input requirements.
The final formatting must then be identical in structure to the manually annotated
detection data supplied with the dataset.

\vspace{3mm}

\textbf{3.4.1.1~~~~Manual Sample}\vspace{4.5mm}\\
When formatting the distance data generated from the manual sample, the goal was to
overwrite each of the manually annotated distances (in original supplied dataframe)
with the corresponding model estimated distances, leaving all other data unchanged.

For reasons detailed in Section~\ref{subsubsec:distance_comparison}, a decision was
made to format the data such as to only overwrite manual distances associated with
frames capturing a single chimpanzee.
This gives a 'supplemented' dataframe, where distance estimates originating from
single-chimp frames are model-estimated while those from multi-chimp frames are estimated
manually.
Nevertheless, density and abundance estimates based on supplemented data are still
informative and give insight into the effectiveness and accuracy of using model estimated
distance data in this context.

\vspace{3mm}

\textbf{3.4.1.2~~~~Automated Sample}\vspace{4.5mm}\\
When formatting the distance data generated from the automated sample

\subsubsection{Script}