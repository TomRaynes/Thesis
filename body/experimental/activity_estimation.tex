\subsection{Activity Estimation}

Using the computed distance estimates from the previous section, chimpanzee population
density and abundance was estimated.
Estimates were calculated for each of the detector/depth model combinations using a
script adapted from one supplied with the dataset.

\subsubsection{Formatting}

For the distance data to be used for density and abundance estimation, it first must be
formatted and joined with the relevant location and camera trap video metadata as to
satisfy the script's input requirements.
The final must then be identical in structure to the manually annotated
detection data supplied with the dataset in CSV formatting containing the columns
'video\_name', 'time\_in\_place\_(days)', 'effort', 'starting\_date', 'date\_observation',
'time\_observation', 'year', 'month', 'day', 'hour', 'minute', 'second', 'distance', 'Sample.Label',
'Effort', 'Region.Label' and 'Area'.

%\begin{itemize}\setlength{\itemsep}{-8pt}
%    \item video\_name
%    \item time\_in\_place\_(days)
%    \item effort
%    \item starting\_date
%    \item date\_observation
%    \item time\_observation
%    \item year
%    \item month
%    \item day
%    \item hour
%    \item minute
%    \item second
%    \item distance
%    \item Sample.Label
%    \item Effort
%    \item Region.Label
%    \item Area
%\end{itemize}

\vspace{3mm}

\textbf{3.4.1.1~~~~Manual Sample}\vspace{4.5mm}\\
When formatting the distance data generated from the manual sample, the goal was to
overwrite each of the manually annotated distances (in original supplied dataframe)
with the corresponding model estimated distances, leaving all other data unchanged.

For reasons detailed in Section~\ref{subsubsec:distance_comparison}, a decision was
made to format the data such as to only overwrite manual distances associated with
frames capturing a single chimpanzee.
This gave a 'supplemented' dataframe, where distance estimates originating from
single-chimp frames were model-estimated while those from multi-chimp frames were
estimated manually.
Nevertheless, density and abundance estimates based on supplemented data are still
informative and give insight into the effectiveness and accuracy of using model estimated
distance data in this context.

\vspace{3mm}

\textbf{3.4.1.2~~~~Automated Sample}\vspace{4.5mm}\\
When formatting the distance data generated from the automated sample however, a
different method was used.
First, a lookup table was created to hold the metadata associated with each specific
camera trap video.
A script was then run to join each of the modelled distances with the video metadata
on video name which gave a formatted dataframe holding entirely model estimated distances.


\subsubsection{Script}